\documentclass[letterpaper, twoside, 12pt]{memoir}
\usepackage{fontspec}
\usepackage{marginnote}

\setmainfont{Junicode}

\chapterstyle{ell}
\pagestyle{Ruled}

\title{{\HUGE Keeper's Call}\\
{\small Game Design Document}
\vspace{\fill} \\ 
Minstrelcy Studios
\vspace{\fill} \\ 
}
\author{J.R.~Omahen}

\begin{document}

\maketitle
\newpage

\tableofcontents

\chapter{Game Design}
\section{Summary}

A girl is wandering through the forest, playing near a lake. She hears a mysterious voice: a call, beckoning her to the water. There she discovers her true calling.

\section{Gameplay}

The player operates in first person, exploring the area surrounding the lake. The mysterious voice will beckon to the player, but the player won't know where it's coming from. The main obstacle to the player is a lack of familiarity with the environment, and not knowing where the voice is coming from. 

\section{Mindset}

The player should be intrigued, seeking to discover what the voice is, where it's coming from, and who it belongs to.

\chapter{Technical}
\section{Summary}

The game is a text-based adventure game (\textit{interactive fiction}), with very simple keyboard-based input. The target play time is \textbf{10---15 minutes}.

\section{Screens}

The opening screen will have an intro title, graphic, and note about the "help" option. The word "help" can be entered at any time to give the player the list of acceptable commands. The player will need to type "Start" when they are ready to begin. 

Each screen following will provide the player with a description of the area, including details about which directions are available to them. Finally, there will be a few areas where the player will "hear something," giving them the option to "listen." These scenes will be important story building points. 

\section{Controls}

Players will be able to enter the following inputs:
\\

\textbf{Directional}
\begin{itemize}
	\item Straight
	\item Left
	\item Right
	\item Turn around \\
\end{itemize} 

\textbf{Actions}
\begin{itemize}
	\item Listen
	\item Don't Listen
	\item Pick up \\
\end{itemize} 

\textbf{Support Menu}
\begin{itemize}
	\item Help
\end{itemize}

\section{Mechanics}
The game has two basic mechanics: \\

\textbf{Start -> Scene Description -> Direction input -> New Scene Description} \\

\textbf{Start -> Scene Description -> Action input -> Action Response -> Direction input -> New Scene Description}

\chapter{Level Design}

\textit{Keeper's Call} features a very simple environment with two thematic elements, and a simple story outline.

\section{Thematic Elements}

\begin{enumerate}
\item The Lake
  \begin{enumerate}
  \item Location
    \begin{enumerate}
    \item Scottish Loch
    \end{enumerate}
  \item Weather
    \begin{enumerate}
    \item Overcast
    \end{enumerate}
  \item Ambience
    \begin{enumerate}
    \item Opaque water
    \item Cool, cold
    \item Misty
    \end{enumerate}
  \end{enumerate}
\item The Forest
  \begin{enumerate}
  \item Location
    \begin{enumerate}
    \item Surrounds The Lake
    \end{enumerate}
  \item Ambience
    \begin{enumerate}
    \item Dark, dim
    \item Rustling noises
    \item Wind or breeze can be heard
    \end{enumerate}
  \end{enumerate}
\end{enumerate}

\section{Story Flow}

\marginpar{\textit{Segment 1}}
“Moira, why don’t you go and play outside?”
That was what her parents always said to her when they gave each other secret looks they didn’t think she could interpret yet. But she could. They meant, \textit{little girls shouldn’t know these things}. Moira didn’t feel little. At nine, she felt very grown-up, certainly much more able to hear something bad---for those secret looks were somber---than her younger brother Duncan. He was only five.

\marginpar{\textit{Segment 2}}
Moira didn’t like the angry black brows that knitted together just below her father’s forehead when she failed to listen, so on this day, otherwise an ordinary one in the village of \textit{Cois an Locha}, she ran out the door---and stopped.
The day was not a fine one. In the morning clouds had merely brooded and threatened, but now they began to leak a sort of whining drizzle. Moira didn’t mind playing in the rain. Her parents didn’t want her to, though, so it must be something especially awful they needed to talk over, something that preoccupied them to the point of forgetfulness.

\marginpar{\textit{Segment 3}}
Moira hesitated underneath the eaves of their cottage. Should she go back inside and risk the black brows of her father? While she stood in indecision, watching the droplets of rain as they plashed onto a patch of grass underneath her feet, voices wafted out to her from the sliver of a window cracked open in the summer warmth.

\marginpar{\textit{Segment 4}}
“There’s been talk of sheep missing from the MacCowan farm. And no trace of blood.” It was her mother who spoke first, quick and anxious.
“That’s just womenfolk tongues, making a fire out of a flicker. We’ve naught to fear of that. Old MacCowan is half-mad with drink, and I’ve no doubt he couldn’t count his own children. I’m more worried about the fields. Harvest is nigh, and somebody’s trampled Gallach’s crop. It’s those lads, I tell you---” That was her father. Low and fierce and growling.
“But it could be---couldn’t it---I mean---Mrs. O’ Cain says she actually saw something in the night---”
Moira’s father laughed, a loud, trumpet-like sound that made her jump. “Hens pecking, the lot! Why, if you believed every green-eyed story that went through this blessed town---” 

\marginpar{\textit{Segment 5}}
Then Moira heard a shuffling of feet, and footsteps thumping towards her. Panicking, she fled out into the drizzle and across the damp grass, her legs carrying her as fast as they could towards the shelter of the forest that lay just before \textit{Druim na Drochaid}. 
She didn’t stop until she was safely at the back of a great oak, heaving and panting, her heart leaping inside of her. She grasped at rough edges of bark as she sank to the ground. Her mother would scold her if she got her dress too dirty, but eavesdropping was a much greater sin than dirt.

\marginpar{\textit{Segment 6}}
More at ease, her breath coming regularly now, Moira looked about her. The wood’s shadows and tangled branches in their familiarity no longer frightened her as they once had. The trees didn’t make her feel quite comfortable---exactly---there was something so unpredictable even now about them, but somehow they did make her feel safe, as if the clustering of branches and the knotty, twisted roots were there to protect and not to forbid her.
Her parents had hurried her out before she could remember to bring her doll, so she sat for a little while, smiling at the thick leafy patches that discouraged the light rain from descending on her. It was pleasant, wasn’t it, to sit beneath a tree, and think anything, or nothing---

\marginpar{\textit{Segment 7}}
\textit{Moira.}
Her heart almost stopped, and then went on, painfully rapid. Had her parents noticed after all that she had been there, listening? She whipped her head round the tree and saw nothing, only the cottage in miniature as it stood on the hill to the west. She sighed in relief.

\marginpar{\textit{Segment 8}}
\textit{Moira.}
There it was again! This time, it was different---or---she heard it differently, as if it had echoed inside her, inside her mind. Moira twitched her ears, confused and excited.

\marginpar{\textit{Segment 9}}
\textit{Moira.} The voice thrummed through her.
“Yes?” She stood, speaking aloud, which seemed silly, but she didn’t know how to do what the voice was doing.
\textit{Come.}

\marginpar{\textit{Segment 10}}
Where? Thought Moira, but she didn’t say it, and she began to think she must be slow-witted not to understand what was going on. Her feet moved reluctantly in one direction and then the next.

\marginpar{\textit{Segment 11}}
\textit{Come}. The tone---if it could be called a tone---was almost impatient.
She turned towards the edge of the forest, towards the bank where the trees thinned, she knew, and a long strip of land lay below before the lake. Perhaps whoever it was who called her name was really just out of sight.

\marginpar{\textit{Segment 12}}
She walked nearer, cautious, and scrambled down the bank, sliding the last quarter of the way down as she slipped in the fresh mud. Her mother would be furious.
Moira.
Good. It was still there---but---where was it? Moira glanced left and right, but all was deserted. No boat was on the lake, and no wave disturbed its glassy calm.

\marginpar{\textit{Segment 13}}
Until she heard something else---a real sound, a definite sound from the outside, that still somehow reverberated through her. It was a crashing, a roaring of the lake. The lake itself seemed to tear and gape open.

\marginpar{\textit{Segment 14}}
All at once, a great shimmery, shiny bulk rose smoothly out of the water, which rippled and scurried away as if in fear. Shale-colored eyes peered down at her.
\textit{Moira. At last.}

\marginpar{\textit{Segment 15}}
Moira gasped. She felt very small and very young. 
“Well.” She shivered. “I know who \textit{you} are.” It was all she could think to say.


\chapter{Development}

Development will happen in two phases: underlying engine and game proper.

\section{Fabler}

\textit{Fabler} is the name of the game engine itself. It is a simple graphically-rendered text engine for interactive fiction. It is written in JavaScript. 

\subsection{Architecture}

\textit{Components} drive the functionality of the engine internals. Each feature will be implemented as component object that can be plugged in to the engine framework. Additional features mean additional components, which can be tested individually.

The text will be rendered graphically onto a \texttt{Canvas}, and input taken by key strokes. Input will be collected by an input listener, then fed to a command parser. The parser will take the input and interpret the instructions. Any changes to the game environment will be requested through the engine by the parser. 

Command parsing will be simple: instructions are one word, and can have one or more arguments afterwards. Each command will correspond to a function that will accept the arguments. 

The game loop will be responsible for orchestrating the mechanics, input and display rendering. The game loop should be supplied by the game implementation, with a simple skeleton structure provided by Fabler.

\subsection{Command Parser}

There will be a simple command parser (outlined above), that will accept input in the following format:

\begin{center}
  \framebox[1.1\width][c]{\texttt{[command] + \textvisiblespace{}  +  ([argument$_1$] \textvisiblespace{}  [argument$_2$] \textvisiblespace{}  ... [argument$_n$])}}
\end{center}

The processing of input looks similar to the following:

\begin{center}
  \framebox[1.1\width]{\textit{Raw text input}} \\
  \vspace{\fill}
  {\large $\downarrow$} \\
  \vspace{\fill}
  \framebox[1.1\width]{\texttt{Parser} $\rightarrow$ \texttt{Tokeniser} --- \texttt{command + [arg$_1$\ldots{}arg$_n$]}} \\
  \vspace{\fill}
  {\large $\downarrow$} \\
  \vspace{\fill}
  \framebox[1.1\width]{\texttt{CommandExecutor} $\rightarrow$ \textit{Execute function} --- \texttt{command(arg$_1$,\ldots{}arg$_n$)}} \\
  \vspace{\fill}
  {\large $\downarrow$} \\
  \vspace{\fill}
  \framebox[1.1\width]{\textit{Game loop increment}} \\
\end{center}

\subsubsection{Dictionary}
The proposed list of commands is as follows. In addition to commands and arguments, there will be necessary tokens to make a phrase or sentence grammatical, but is irrelevant to the command itself. These will be removed during parsing, but available to the end user.

\begin{center}
  \begin{tabular}{c} 
    \toprule
    \textit{Commands} \\
    \midrule
    \texttt{go} \\
    \texttt{look} \\
    \texttt{take} \\
    \texttt{open} \\
    \texttt{read} \\
    \texttt{talk} \\
    \texttt{listen} \\
    \bottomrule
  \end{tabular}
  \hspace{\fill}
  \begin{tabular}{c} 
    \toprule
    \textit{Directions} \\
    \midrule
    \texttt{forward} \\
    \texttt{back} \\
    \texttt{left} \\
    \texttt{right} \\
    \bottomrule
  \end{tabular}
  \hspace{\fill}
  \begin{tabular}{c} 
    \toprule
    \textit{Tokens} \\
    \midrule
    a(n) \\
    the \\
    of \\
    to \\
    in \\
    on \\
    for \\
    with \\
    ---(')s \\
    \bottomrule
  \end{tabular}

\end{center}

\subsection{Components}
\begin{itemize}
\item Game (FABLER)
\item Module
\end{itemize}
\subsection{Component Compositions}
\begin{itemize}
\item Module
  \begin{itemize}
  \item GfxMan
  \item Screen
  \item Player
  \item Location
  \item World
  \item Input
  \item Parser
  \item Event
  \item Item
  \end{itemize}
\end{itemize}

\section{Game Program}

\subsection{Application Container}

The game will run on the \textit{Fabler} engine, encapsulated in an embedded \textit{Chromium} instance, and will be written in JavaScript.

\subsection{Startup}

Upon launch, the game will run in full screen\footnote{This may requiring prompting the user with a dialogue confirmation}, presenting itself as on old terminal console.

\subsection{Game Loop}

The game loop is managed by \textit{Fabler}, but the creation of all assets, story points and game flow will be hard--coded into the game program for simplicity. We will not be concerned with portability or ease of modification.

\subsection{Shutdown}

The game will exit by either the successful completion of the story, or the player entering an appropriate command in the terminal.

\chapter{Development Timeline}

Total project estimate is \textbf{30 hours}.

\section{September 2015}

\begin{itemize}
\item Game concept is agreed upon, and pre--production begins
\item Brainstorming results in the majority of game design outlining 
\item GDD begun and developed
\end{itemize}

\section{October 2015}

\begin{itemize}
\item Pre--production grinds to a halt as team members are distracted with illness, work and other pursuits
\item Some amount of tinkering with the GDD and server ensues
\end{itemize}

\section{November 2015}

\begin{itemize}
\item Pre--production resumes with the finalising of the GDD
\end{itemize}

\section{December 2015}

Starting with this month, development time is assumed to be \textbf{1-3 hours per week}.

\begin{itemize}
\item Production begins on the Fabler engine
\item Production begins on the box an manual art
\item Production begins on the story assets
\item Some early testing of the running game engine begins
\end{itemize}

\section{January 2016}

\begin{itemize}
\item Testing begins of the game in the engine
\item Final assets assembled into game
\item Full game is tested and refined
\end{itemize}

\section{February 2016}
\begin{itemize}
\item Game is released to manufacturer
\item Marketing material is created
\item Manuals are created
\item Final packaging completed
\item Game published
\end{itemize}
\chapter{Proposals}

Please place proposals in new sections with any relevant and necessary information. If a proposal is adopted, ensure that it is added/incorporated into the relevant section. If discarded, please add it to the \textit{Rejected Ideas} chapter, along with the discussion or reasoning.

\section{Brainstorming Notes}

These represent the areas agreed upon. Extra brainstorming notes available in the documents for Keeper's Call.

\begin{itemize}
\item Girl is the keeper of the Loch Ness monster
\item Game is entitled \textit{Keeper's Call}
\item Nessie calls out. We want to avoid something inherent about the girl that makes Nessie want to reach out. We can allow readers to believe she is worthy, but gradually her flaws are revealed
\end{itemize}

\chapter{Rejected Ideas}

\section{Hergen, Bergen, Fergen}

With an option on \textit{Dergen}. 

\end{document}